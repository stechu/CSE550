\documentclass[11pt, onecolumn]{article}
\usepackage[pdftex]{graphicx}
\usepackage{wrapfig}
\usepackage{enumerate}
\usepackage{hyperref}

\begin{document}

\title{CSE550 Problem Set 2: Paxos Design Document}
\author{Vincent Lee, Shumo Chu}
\date{\today}

\maketitle

\tableofcontents

%%%%%%%%%%%%%%%%%%%%%%%%%%%%%%%%%%%%%%%%%%%%%%%%%%%%%%%%%%%%%%%%%%%%%%%%%%%%%%
% Intro
%%%%%%%%%%%%%%%%%%%%%%%%%%%%%%%%%%%%%%%%%%%%%%%%%%%%%%%%%%%%%%%%%%%%%%%%%%%%%%

\section{Introduction}

In this document, we outlined our software implementation of the Paxos consensus protocol explained in TODO: ADD REFERENCE.
We present a high level software architecture, justify our design decisions, and explain any simplfying assumptions that we make to our design.
We then explain how to use our implementation, and any outstanding problems with our implementation.
A quick discussion of any interesting tidbits we encounter concludes the document.

%%%%%%%%%%%%%%%%%%%%%%%%%%%%%%%%%%%%%%%%%%%%%%%%%%%%%%%%%%%%%%%%%%%%%%%%%%%%%%
% Paxos Protocol Summary
%%%%%%%%%%%%%%%%%%%%%%%%%%%%%%%%%%%%%%%%%%%%%%%%%%%%%%%%%%%%%%%%%%%%%%%%%%%%%%

\section{Paxos Protocol}

The Paxos protocol originally proposed by Lamport in TODO: ADD REFERENCE is a consensus protocol designed to make forward progress even in the face of arbitrary failures.
We interpret consensus to be defined as the following constraints:

\begin{enumerate}
\item One and only one value may be chosen per Paxos instance
\item The value must come from the set of proposals
\item Only the chosen values can propagate to learners
\end{enumerate}

The Paxos protocol consists of three classes of members: proposers, acceptors, and learners.
Proposers are defined as entities that present proposals to acceptors.
Acceptors are defined as entities that are responsible for arbitrating whether a proposal is accepted or not.
Finally, learners are defined as entities that are subscribed to a particular value that a consensus determines.

In any Paxos instance, in order to tolerate node failure, multiple proposers and acceptors participate in a Paxos instance.
A Paxos instance is resolved by the following a two phase protocol.

In the first phase of the protocol, proposers create proposals with some associated monotonically increasing sequence number N. 
The sequence number is always larger than any previous sequence number the proposer has issued. 
A propose message is then sent to the set of acceptors with the sequence number chosen. 
When an acceptor receives a propose message, it compares the sequence number of the proposal to all previous sequence numbers it has seen.
If the sequence number received is less than any sequence number the acceptor has seen, the proposal is rejected and ignored.
If the sequence number received by the acceptor is greater than or equal to the prior sequence numbers observed by the acceptor, it ``promises'' to reject all messages less than the sequence number received from that point forward.
A response is issued with this ``promise'' to the original proposer.

Meanwhile, the original proposer wait until it receives responses from a majority of the acceptors it initially issued proposals to with sequence number N.
Once this occurs, the proposer issues another message with the value chosen for the proposal that was issued.
An acceptor will accept the proposal and chosen value if the highest sequence number is sees is still N, and reject the message otherwise.
Upon successful receipt of the accept message, messages are issued to all relevant learners with the value chosen and an accept message is issued to the original proposer.
If an acceptor rejects the message, the message will be discarded and no response will be issued.
A proposer knows its attempt to chose a value has failed if it does not receive responses from a majority of acceptors.

If at the end of a single iteration of this protocol, no value is ultimately chosen, the procedure is repeated with a higher sequence number of each node until a value is eventually chosen.

%%%%%%%%%%%%%%%%%%%%%%%%%%%%%%%%%%%%%%%%%%%%%%%%%%%%%%%%%%%%%%%%%%%%%%%%%%%%%%
% Implementation Details
%%%%%%%%%%%%%%%%%%%%%%%%%%%%%%%%%%%%%%%%%%%%%%%%%%%%%%%%%%%%%%%%%%%%%%%%%%%%%%

\section{Paxos Implementation}

We assume the same assumptions about the distributed system model as originally proposed in the Paxos paper. Namely, we assume the following:

\begin{enumerate}
\item No Byzantine failures will occur.
\item All entities in the distributed system operate asynchronously a arbitrary speed
\item Any node that fails in the system will observe fail-stop behavior
\item Messages at any phase during the protocol can be dropped, reordered, or duplicated
\end{enumerate}

\subsection{Simplifying Assumptions}

We also make the following assumptions to simplify our implementation of the Paxos algorithm:

\begin{enumerate}
\item Any nodes that experience a failure will permanently be removed from the Paxos group and will not attempt recovery and restart
\item Clients to the system will not intentionally behave maliciously
\item Entities will not attempt to retransmit messages in the case of message loss or timeout
\item Clients know which entities are part of the Paxos group
\item The client-server communication channel is reliable
\end{enumerate}

TODO: figure out how to elect a leader

\subsection{High Level Software Architecture}

STUB


%%%%%%%%%%%%%%%%%%%%%%%%%%%%%%%%%%%%%%%%%%%%%%%%%%%%%%%%%%%%%%%%%%%%%%%%%%%%%%
% Instructions for Use
%%%%%%%%%%%%%%%%%%%%%%%%%%%%%%%%%%%%%%%%%%%%%%%%%%%%%%%%%%%%%%%%%%%%%%%%%%%%%%

\section{How to Use Our Implementation}

STUB

%%%%%%%%%%%%%%%%%%%%%%%%%%%%%%%%%%%%%%%%%%%%%%%%%%%%%%%%%%%%%%%%%%%%%%%%%%%%%%
% Known Issues
%%%%%%%%%%%%%%%%%%%%%%%%%%%%%%%%%%%%%%%%%%%%%%%%%%%%%%%%%%%%%%%%%%%%%%%%%%%%%%

\section{Known Issues}

STUB

%%%%%%%%%%%%%%%%%%%%%%%%%%%%%%%%%%%%%%%%%%%%%%%%%%%%%%%%%%%%%%%%%%%%%%%%%%%%%%
% Discussion
%%%%%%%%%%%%%%%%%%%%%%%%%%%%%%%%%%%%%%%%%%%%%%%%%%%%%%%%%%%%%%%%%%%%%%%%%%%%%%

\section{Discussion}

STUB

\end{document}
